\chapter{Code Structure and I/O}
%
%
\section{Files for Data Input}
%
%
The \emph{Goma} file I/O structure is diagrammed in Figure 2. Input to the program is divided into six categories: (1) command-line options, (2) problem description file, (3) material files, (4) ASCII continuation/restart file, (5) EXODUS II database file, and (6) sundry material property or boundary condition table lookup files. \emph{Goma} is basically set up to run in batch mode, i.e., no input is required on the command line or after the run command is issued. There are, however, several command-line switches which can be used to redirect I/O, control the level of I/O, and activate debugging options.

The \emph{problem-description file} is by default called “input” but can be renamed with the -\textbf{i} switch on the command line. A version of this file is also output as an “echo” file, viz. a prefix “echo” prepended to the input file name. The echo file is used to verify input into goma, as it clearly
states all default settings for the input file and material files. . The input file itself contains the general description of the problem and directions to Goma on how to solve it (see Chapter 4). The file is split into fifteen sections: (1) File Specifications (Section 4.1) which directs I/O, (2) General Specifications (Section 4.2), (3) Time Integration Specifications (Section 4.3), (4) Continuation Specifications (Section 4.4), (5) Hunting Specifications (Section 4.5), (6) Augmenting Condition Specification (Section 4.6), (7) Solver Specifications (Section 4.7), (8) Eigensolver Specifications (Section 4.8), (9) Geometry Specification (Section 4.9), (10) Boundary Condition Specifications (Section 4.10), (11) Rotation Specifications (Section 4.11), (12) Problem Description (Section 4.12), and (13) Post Processing Specifications (Section 4.13); this latter section includes breakouts for fluxes and data (Section 4.14), particle traces (Section 4.15) and for volume-based integrals. The file format is described in detail in Chapter 4. Incidentally, the structure of the data input routines is divided roughly along the same lines as the input data file itself.

The \emph{material description} files (using the nomenclature “[material name].mat”) contain all material property data and material property model and constitutive model specifications. The names of these files are specified in the problem description file. The format of these files and the available options are described in Chapter 5. Note that these files are also reproduced as output as “echo” files, with all default settings specified.

The \emph{ASCII continuation/restart files} (may have any name) contain an ASCII list of the solution vector (values of field variables at nodes), which can be used as an initial guess for successive runs of \emph{Goma}. The names of these files are specified in the problem description file, but may be changed with the -\textbf{c} (for input) or -\textbf{s} (for output) command-line options. These restart files are “recyclable”, in the sense that output from one \emph{Goma} simulation may be used as input to another \emph{Goma} simulation under certain restrictions.

The \emph{EXODUS II database files} (may have any name but generally end in “.exoII”) contain a description of the finite-element structure for the current problem. All EXODUS II files contain a definition of the mesh, material blocks, and boundary sets. In the case of input EXODUS II files created from mesh generator output, this is the sole content of the file. Output EXODUS II database files contain a clone of the input EXODUS II mesh information and also contains the nodal values of all field variables in the solution. The names of these files are specified in the problem description file, but may be changed with the -\textbf{ix} (for input) or -\textbf{ox} (for output) command-line options. The only EXODUS II file required when running Goma is the one containing the current problem mesh. All others are either output for postprocessing or used to supply auxiliary external fields (e.g. magnetic fields).